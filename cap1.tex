\chapter{Introdução}\label{intro}
\section{Primeira seção da introdução}\label{intro:historico}
% O comando \label{nome} define o marcador da parte especificada.
% Você pode citar esta seção usando o comando \ref{nome}.
% O "~" evita uma quebra de linha entre as palavras.

O Capítulo~\ref{intro} é uma introdução ao contexto do projeto.
Aqui vou exemplificar alguns comandos básicos e úteis para uma dissertação como incluir citações \citep{Sand-Jensen2007}.

% Veja mais formas de fazer citações no texto da documentação do natbib.
O \texttt{natbib} é um ótimo pacote, pois é flexível \citep[ver detalhes em][]{Kirk2008}.
\citet{Emlet1987} mostra outro modo de citar trabalhos no texto e como grafar o nome das espécies \emph{Drosophila~melagonaster} e \emph{Clypeaster~subdepressus}.
% Neste caso o uso do "~" foi por opção.

\section{Segunda seção da introdução}\label{intro:contexto}

Mencionei na seção~\ref{intro:historico} como citar um capítulo.
O modelo está dividido em 2 capítulos, sendo que o Capítulo~\ref{cap2} trata de um assunto (e.g., um artigo decorrente da sua pesquisa) enquanto o Capítulo~\ref{cap3} aborda outro.

% O comando abaixo faz com que uma página em branco seja incluída após o fim do capítulo.
% Use-o caso você precise criar uma página em branco para que os capítulos se iniciem sempre em páginas ímpares.
% Também é útil para deixar as páginas certas e contadas para a impressão.

% \clearpage{\pagestyle{empty}\cleardoublepage}
