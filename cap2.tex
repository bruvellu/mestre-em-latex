\chapter{Um assunto legal}\label{cap2}

\section{Introdução}\label{cap2:intro}

O modelo novo sugere que cada capítulo tenha um resumo, introdução, materiais e métodos, resultados, discussão e conclusões, mas optei por deixar o resumo de lado.

\section{Materiais e Métodos}\label{cap2:mem}

\subsection{Coleta dos organismos}\label{cap2:mem:coleta}

Você pode dividir cada seção em subseções para organizar melhor o conteúdo.

\subsection{Cultivo dos seres vivos}\label{cap2:mem:gametas}

\subsubsection{Embriões e Larvas}

Você também pode criar subsubseções caso necessário.
% Para inserir fórmulas matemáticas bonitinhas coloque $entre$.
% Note como formatar a unidade de temperatura e como utilizar o comando \nicefrac para frações.
A cultura foi mantida num ciclo de $12:12$ horas e temperatura constante de \unit[24]{°C} e a concentração final entre $8\times10^5$ e $1\times10^6$ \nicefrac{células}{mL}.

\subsubsection{Metamorfose}

% Como o % representa um comentário e não é compilado, para fazê-lo aparecer no texto você precisa colocar uma "\" antes, como abaixo.
Mais uma subsubseção só pelo prazer de escrever subsubseção \citep{Highsmith1982}. Apenas \unit[4]{\%} do texto está contido em subsubseções.

\subsection{Microscopia de luz}\label{cap2:mem:micro}

As amostras também foram colocadas em placas $35\times\unit[10]{mm}$ para visualização na lupa.
Alguns videogramas foram capturados diretamente da câmera de vídeo conectada ao computador via porta \emph{Firewire}.

\section{Resultados}\label{cap2:res}

\subsection{Primeiras figuras}\label{cap2:res:figs}

Subseções de novo, mas agora é hora de colocar algumas figuras para mostrar resultados. Podemos começar com uma bela migração de pró-núcleos de \subde\ (Figura~\ref{fig:pronuc}).

\begin{figure}[htbp]
  \centering
  \includegraphics[width=418pt]{pronuc}
  \caption[Migração dos pró-núcleos]{Montagem mostrando a migração e fusão dos pró-núcleos feminino (verde) e masculino (vermelho) durante \unit[12]{min}~\unit[30]{s}.}
  \label{fig:pronuc}
\end{figure}

Essa pode ser a multifig (Figura~\ref{fig:tufo}).

\begin{figure}[htbp]
  \centering
  \includegraphics[width=300pt]{tufo}
  \caption[Prisma sob MEV]{Larva de \subdeshort\ no estágio de prisma. Na porção anterior encontra-se o tufo apical (\textbf{ta}) e a depressão onde será formada a boca (\textbf{bo}). Na região posterior crescem os braços pós-orais (\textbf{po}) e o futuro ânus (\textbf{an}).}
  \nomenclature{bo}{boca}%
  \nomenclature{ta}{tufo apical}%
  \nomenclature{po}{braços pós-orais}%
  \nomenclature{an}{ânus}
  \label{fig:tufo}
\end{figure}

\subsection{Tabelas}\label{cap2:res:tabs}

Os dados do desenvolvimento de \subdeshort\ da fecundação até a eclosão estão resumidos na Tabela~\ref{tab:exemplo}.

\begin{table}[htbp]
  \caption[Tabela com \texttt{booktable}]{Exemplo de legenda de tabela criada com o pacote \texttt{booktable}.}
  \label{tab:exemplo}
  \vspace{1em}
  \centering
  \begin{tabular}{l l}
    \toprule
    Eventos		&	Tempo\\
    \midrule
    Entrada		&	0\\
    Elevação		&	\unit[40]{s}\\
    Corrida		&	\unit[6]{min}\\
    Saída		&	\unit[15]{min}\\
    \bottomrule
  \end{tabular}
\end{table}

% Usando o pacote units
As duas primeiras são meridionais, isto é, ocorrem ao longo do eixo \nicefrac{animal}{vegetal} (\nicefrac{A}{V}) e dividem o embrião em quatro blastômeros de tamanhos iguais à uma velocidade de \unitfrac[500]{µm}{s}.
A blástula, composta por uma camada de células (ectoderme), passa por processos cruciais de seu desenvolvimento por volta de \unit[7,5]{h} após a fertilização.

% Usando o pacote nomencl
Cerca de \unit[10]{h} após a fertilização, as células mesenquimais primárias (CMP) iniciam sua ingressão na blastocele a partir do pólo vegetativo.%
\nomenclature{CMP}{células mesenquimais primárias}
Nesse momento, é possível flagrar as primeiras células mesenquimais secundárias (CMS) na blastocele.%
\nomenclature{CMS}{células mesenquimais secundárias}

% Colocando aspas e exemplo do pacote natbib
Larvas com rudimento bem desenvolvido apresentam um comportamento de ''teste de substrato`` tocando o substrato com os braços larvais.
Quatro pedicelárias oficéfalas \citep[][pg.~430]{Hyman1955} surgem na região posterior na altura do âmbito, sendo duas posicionados mais lateralmente.

\section{Discussão}\label{cap2:disc}

\citet{Chia1977} verificou que as papilas genitais de \emph{Arachnoides placenta} não têm tecido muscular ou nervoso, sugerindo que a pressão hidrostática dos gametas seja responsável pela elevação das mesmas, como parece ocorrer em \subdeshort.
Ainda não está claro se a formação de agregados é um comportamento reprodutivo de equinodermos ou se está relacionada à proteção e forrageamento dos indivíduos de uma população \citep{Pearse1991}.

A velocidade de migração dos pró-núcleos de \subdeshort\ (\unitfrac[0,1]{µm}{s}) foi praticamente a mesma encontrada em \emph{C.~japonicus} (\unitfrac[0,082]{µm}{s}) e \emph{L.~variegatus} (\unitfrac[0,083]{µm}{s}).
Contudo, não existe qualquer evidência direta de tal relação e não encontramos descrições semelhantes na literatura revisada \citep{Hamaguchi1980,Schatten1981,Pearse1991,Gilbert1997}.
\emph{Mellita quinquiesperforata} desenvolve-se mais rapidamente; eclode \unit[4]{hpf}, completa a gastrulação \unit[7]{hpf} e inicia sua alimentação em \unit[24]{h} \citep{Caldwell1972}\footnote{A temperatura utilizada neste trabalho não foi precisada; os embriões foram fecundados e desenvolveram-se entre 25 e \unit[28]{°C}.}.

A evolução deste caráter pode ser vista de duas formas.

\begin{enumerate}
  \item{Múltiplos esferídios $\longrightarrow$ Um esferídio}\label{hipo:1}
    \begin{itemize}
      \item{Um esferídio seria plesiomórfico para Clypeasteroida}
      \item{Dois esferídios seria uma autapomorfia de Clypeasterina}
    \end{itemize}
  \item{Múltiplos esferídios $\longrightarrow$ Dois esferídios $\longrightarrow$ Um esferídio}\label{hipo:2}
    \begin{itemize}
      \item{Dois esferídios seria plesiomórfico para Clypeasteroida}
    \end{itemize}
\end{enumerate}

Segundo \citet{Mooi1990}, não existem critérios baseados em grupos externos que permitam polarizar o estado do caráter de múltiplos esferídios, para dois esferídios, e para um esferídio [\ref{hipo:2}], apesar de fazer sentido biologicamente.
Uma alternativa seria considerar o esferídio único (Laganina e Scutellina) como plesiomórfico para Clypeasteroida [\ref{hipo:1}], tendo como grupo externo o fóssil \emph{Togocyamus}, que apresenta um único esferídio aberto \citep{Mooi1990}.

Jovens pós-metamórficos de \emph{M.~quinquiesperforata} cresceram de \unit[0,35]{mm} para \unit[3,80]{mm} do primeiro ao segundo mês, chegando a \unit[6,20]{mm} no sexto mês de vida após a metamorfose \citep{Caldwell1972}.

\clearpage{\pagestyle{empty}\cleardoublepage}
